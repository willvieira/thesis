% ATTENTION 
% ce modèle a été compilé avec la ligne de commande et le script «bashcommandeCompilation.sh» inclus dans le dossier These. 
% Ceci a grandement facilité la compilation des différents chapitres et leurs bibliographiesrespectives. 
% Sinon il faut s'assurer de compiler manuellement chacun des chapitres et de rouler bibtex pour chacun d'eux, ceci dans le même ordre que spécifié dans le script bash. 
% Ce modèle répond aux exigences du Département de Biologie en date de Février 2021. 
% Ce code produit TheseRef.pdf qui inclut chaque bibliographie par chapitre et la bibliographie générale. 

\documentclass[12pt,oneside]{book}
\usepackage[a-1b]{pdfx}
\usepackage[T1]{fontenc}
\usepackage[utf8]{inputenc}
\usepackage{lipsum}

%\includeonly{chapitre1}  	%pour inclure seulement quelques chapitres dans la compilation (et faire des tests)


%-------------------LANGUE ET BIBTEX----------------------------
\usepackage[a4paper]{geometry}
\usepackage{mathpazo} 			% utilise Palatino pour les mathématiques (mettre en premier)
\usepackage{newtxtext, newtxmath} 	% utilise la police Times Roman
\usepackage[sectionbib]{natbib}	% avant BABEL
\usepackage{chapterbib}			% avoir plusieurs bibliographies dans le même document avec la commande \include 
\usepackage[french]{babel}  		% comment this line if the thesis is in English
\usepackage{hyperref}
\setlength{\bibhang}{0pt}			% aucune indentation dans les bibliographies  
\setlength{\bibsep}{10pt}			% 1 interligne et demi entre les entrées de la bibliographie
%%% Uncomment if you want to include the bibliographies at the end of each chapter in the table of contents.  
% \usepackage[nottoc]{tocbibind}
\usepackage{longtable,booktabs}


%-------------------FORMAT----------------------------
\usepackage{lineno}			% ajoute des numero de ligne (pour PREMIER DÉPÔT SEULEMENT - COMMENTER POUR DÉPÔT FINAL)
\usepackage{pdflscape}		% permet d'avoir des page horizontale (utile pour grande table large)
\usepackage[section]{placeins}	% ajoute la commande \FloatBarrier qui empêche les figure de trop bouger	
\usepackage{graphicx}		% gère l'insertion des figures
\makeatletter
\def\maxwidth{\ifdim\Gin@nat@width>\linewidth\linewidth\else\Gin@nat@width\fi}
\def\maxheight{\ifdim\Gin@nat@height>\textheight\textheight\else\Gin@nat@height\fi}
\makeatother
% Scale images if necessary, so that they will not overflow the page
% margins by default, and it is still possible to overwrite the defaults
% using explicit options in \includegraphics[width, height, ...]{}
\setkeys{Gin}{width=\maxwidth,height=\maxheight,keepaspectratio}
% Set default figure placement to htbp
\makeatletter
\def\fps@figure{htbp}
\makeatother

\usepackage{setspace} 		% gère l'interligne
\usepackage{amsfonts}		% ajoute des polices mathématiqueS
\usepackage{amsmath}             % ajoute des environnements mathématiques
\usepackage{mathrsfs}		% ajoute une meilleure police calligraphique pour certains symboles
\usepackage{gensymb}		 % gère les symboles
\usepackage{xcolor}


%% page en format paysage avec numéro de page en bas, centré 
\usepackage{everypage}
\newlength{\hfoot}
\newlength{\vfoot}
\AddEverypageHook{\ifdim\textwidth=\linewidth\relax
\else\setlength{\hfoot}{-\topmargin}%
\addtolength{\hfoot}{-\headheight}%
\addtolength{\hfoot}{-\headsep}%
\addtolength{\hfoot}{-.5\linewidth}%
\ifodd\value{page}\setlength{\vfoot}{\oddsidemargin}%
\else\setlength{\vfoot}{\evensidemargin}\fi%
\addtolength{\vfoot}{\textheight}%
\addtolength{\vfoot}{\footskip}%
\raisebox{\hfoot}[0pt][0pt]{\rlap{\hspace{\vfoot}\rotatebox[origin=cB]{90}{\thepage}}}\fi}

%-------------------FIGURES ET TABLEAUX ----------------------------
%\usepackage{subcaption}		 % pour mettre des figures côte à côte 
\usepackage{xstring}
\usepackage{etoolbox}
\usepackage[labelsep=none]{caption}

% FORMAT CAPTION
% Première phrase en gras et le reste dans un paragraphe en dessous
\newcommand\firstsentencebold[1]{%
    \noexpandarg
    \exploregroups
    \IfSubStr{#1}{. }{%
        \StrBefore{#1}{. }[\firstcaptionsentence]%
        \StrBehind{#1}{. }[\othercaptionsentences]%
        \textbf{\firstcaptionsentence. }\smallskip\othercaptionsentences%
    }{%
        \textbf{#1.}%
    }%
}

\makeatletter
% Custom caption format based on the original hang format with first sentence bold
\DeclareCaptionFormat{customhang}{%
  \caption@iflabelseparatorwithnewline
    {\caption@Error{%
       The option `labelsep=\caption@labelsep@name' does not work\MessageBreak
       with `format=customhang'}}%
    {\@hangfrom{\textbf{#1#2 }}%
     \advance\caption@parindent\hangindent\relax
     \advance\caption@hangindent\hangindent\relax
     \caption@@par\firstsentencebold{#3}}}

\captionsetup{format=customhang}

\usepackage{epsfig} 			% ajouter et convertir figures .eps. Ne pas mettre l'extension dans le document 
\usepackage[most]{tcolorbox}
\usepackage[flushleft]{threeparttable} % notes de bas de tableau 
\usepackage{adjustbox}			 % gère les gros tableaux 
\usepackage{tabularx, multirow,booktabs}

%------------------COMPILATION PDFLATEX----------------------------
\newif\ifhyper\hypertrue  	% options PDF. Requiert de compiler avec pdflatex
%\hyperfalse 			% décommenter pour supprimer les hyperliens (version imprimée)
\ifhyper\usepackage[pdfa]{hyperref} 
\urlstyle{same}
\hypersetup{ 
     backref=true, pagebackref=true,   % ajoute les liens dans la bibliographie
     hyperindex=true, 		% ajoute des liens dans les index. 
     colorlinks=true, 		% colore les liens 
     breaklinks=true,		% permet le retour la ligne dans les liens trop longs 
     urlcolor= black,  		% couleur des hyperliens (doit inclure x11names dans xcolor ci-dessus)
     linkcolor= black, 		% couleur des liens internes 
     citecolor=black,		% couleur des liens de citation
     bookmarks=true,		% créationŽŽ des signets PDF 
     bookmarksopen=true,	% ouvre les signets PDF au départ 
}\else\relax\fi

%-------------------PAGINATION ----------------------------

\makeatletter			% ENLÈVE LA PAGINATION DES PREMIÈRES PAGES DE SECTIONS
\renewcommand\chapter{\if@openright\cleardoublepage\else\clearpage\fi
                    \thispagestyle{empty}%
                    \global\@topnum\z@
                    \@afterindentfalse
                    \secdef\@chapter\@schapter}
\makeatother


%------------------FORMAT DES CHAPITRES----------------------------
\pagestyle{plain} 		% Entêtes et pieds % pas d'entête, no de page en bas
\usepackage{titlesec}	% Format de titres de chapitres, sections et sous-sections
\usepackage[skip=24pt]{parskip} % Vertical skip entre les paragraphes
\newcommand\chapterstring{CHAPITRE}

%\titleformat{command to change}[shape of the title]{format of the title}{label of the title}{space between label and title}{code preceding the title body}[code following the title body]
%\titlespacing*{command to change}{left space}{space with paragraph before the title}{space with paragraph after the title}[right space]


% Règle 2.4.5 
% Entre la numérotation du chapitre et le titre du chapitre, introduire un changement de ligne (1 ½ interligne).
\titleformat{\chapter}[display]{\vspace{-6em}\bfseries\center}{\chapterstring~\thechapter}{0pt}{}
\titleformat{\section}{\normalfont\bfseries}{\thesection}{10pt}{}
\titleformat{\subsection}{\normalfont\bfseries}{\thesubsection}{10pt}{}
\titleformat{\subsubsection}{\normalfont\normalsize\bfseries}{\thesubsubsection}{10pt}{}
\titleformat{\paragraph}[runin]{\normalfont\normalsize\bfseries}{\theparagraph}{10pt}{}
\titleformat{\subparagraph}[runin]{\normalfont\normalsize\bfseries}{\thesubparagraph}{10pt}{}


% Après le titre d'un chapitre, le premier titre ou paragraphe du dit chapitre se trouvera à quatre interlignes et demi du titre (36 pts).
\titlespacing*{\chapter}{0pt}{36pt}{36pt} 

% Trois interlignes (ou 24 pts) séparent les titres des sous-titres ou des paragraphes, les sous-titres des paragraphes, les paragraphes des sous-titres et les paragraphes entre eux.
\titlespacing*{\section}{0pt}{24pt}{24pt}
\titlespacing*{\subsection}{0pt}{24pt}{24pt}
\titlespacing*{\subsubsection}{0pt}{24pt}{24pt}
\titlespacing*{\paragraph}{0pt}{24pt}{24pt}
\titlespacing*{\subparagraph}{0pt}{24pt}{24pt}


%------------------MARGE DU DOCUMENT ---------------------------

\geometry{letterpaper,lmargin=1.0in,rmargin=1in,tmargin=1.5in,bmargin=1.0in}
\setlength{\parindent}{0ex} 	% indentation au début de chaque paragraphe
%\setlength{\parskip}{3ex plus 0.3ex minus 0.1ex} % espace vertical entre paragraphes
%\setlength{\parskip}{12pt}
\usepackage{nowidow}


%-------------------SAUTS DE PAGE, FORMAT TABLE DES MATIÈRES ET LISTES ----------------------------
\newcommand{\blankpage}{	% page blanche au tout début du document 
\newpage
\thispagestyle{empty}
\mbox{}
\newpage
}
% table des matières
\usepackage[titles]{tocloft} 	 
\usepackage{calc} 
\renewcommand{\cftchapleader}{\cftdotfill{\cftdotsep}} % Ligne pointillée entre titre de chapitre et # de page


% ceci personnalise la table des matières avec les noms de chapitre, leur numéro ainsi que leur titre, incluant un "-" 
% une macro fait la même chose aux appendixes 
\renewcommand{\cftchappresnum}{CHAPITRE\space}
\setlength{\cftchapnumwidth}{\widthof{\textbf{Appendix~999~}}}
\renewcommand{\cftchapaftersnum}{  -- }
\makeatletter
\g@addto@macro\appendix{%
  \addtocontents{toc}{%
    \protect\renewcommand{\protect\cftchappresnum}{ANNEXE\space}%
        \protect\renewcommand{\protect\cftchapaftersnum}{}%

  }%
}

% pour les annexes : ne pas inclure les numéros de sections dans la TOC mais garder la numérotation pour les figures 
\appto\appendix{\addtocontents{toc}{\protect\setcounter{tocdepth}{0}}}

% reinstate the correct level for list of tables and figures
\appto\listoffigures{\addtocontents{lof}{\protect\setcounter{tocdepth}{1}}}
\appto\listoftables{\addtocontents{lot}{\protect\setcounter{tocdepth}{1}}}


% format liste de figures
\newlength{\mylen}
\renewcommand{\cftfigpresnum}{Figure\enspace}
\renewcommand{\cftfigaftersnum}{\hspace{1mm}}
\settowidth{\mylen}{\cftfigpresnum\cftfigaftersnum}
\addtolength{\cftfignumwidth}{\mylen}

%format liste de tableaux
\newlength{\mylent}
\renewcommand{\cfttabpresnum}{Table\enspace}
\renewcommand{\cfttabaftersnum}{\hspace{1mm}}
\settowidth{\mylent}{\cfttabpresnum\cfttabaftersnum}
\addtolength{\cfttabnumwidth}{\mylent}


\newcommand*{\noaddvspace}{\renewcommand*{\addvspace}[1]{}}
\addtocontents{lof}{\protect\noaddvspace}	% diminue l'espace entre les items de la liste 
\addtocontents{lot}{\protect\noaddvspace}


%------------------CORPS DU DOCUMENT----------------------------

\begin{document}
\pagenumbering{gobble}
%\raggedbottom 	% éviter un étirement bizarre
\blankpage
\blankpage


%-------------------------------------------------------------------------------
%  FRANCISER LES TERMES DE LA PRESENTATION
%-------------------------------------------------------------------------------
\renewcommand{\figurename}{Figure}
\renewcommand{\tablename}{Table} 		% TABLEAU EN FRANÇAIS 
\renewcommand{\chaptername}{CHAPITRE} 
\renewcommand{\contentsname}{TABLE DES MATIÈRES}
\renewcommand{\listtablename}{LISTE DES TABLEAUX}
\renewcommand{\listfigurename}{LISTE DES FIGURES}


%------------------PAGE TITRE----------------------------

\singlespacing
\begin{center}
{
%\sffamily 	% commenter cette ligne pour un lettrage en roman, avec sérif
{\textbf{Effets du climat, de la compétition et de l'emménagement forestier sur les limites de l'aire de répartition des arbres : de l'individu à la métapopulation}} % titre
\\  \vspace{2.5cm}
par
\\   \vspace{2.5cm}
{\textbf{Willian Vieira}} % inscrire votre nom
\\   \vspace{2.5cm}
Thèse présentée au Département de biologie en vue\\
de l'obtention du grade de docteur ès sciences (Ph.D.)\\
\vfill
FACULTÉ DES SCIENCES\\
UNIVERSITÉ DE SHERBROOKE\\  \vspace{1.5cm}
\vfill
Sherbrooke, Québec, [mois et année du dépôt final]
}

\end{center}


\singlespacing
\begin{center}
{
%\sffamily 	% commenter cette ligne pour un lettrage en roman, avec sérif
{\textbf{Effets du climat, de la compétition et de l'emménagement forestier sur les limites de l'aire de répartition des arbres : de l'individu à la métapopulation}} % titre
\\  \vspace{2.5cm}
by
\\   \vspace{2.5cm}
{\textbf{Willian Vieira}} % inscrire votre nom
\\   \vspace{2.5cm}
Dissertation presented to the Biology Depatment\\
for the degree of Doctor of Science (Ph.D.)\\
\vfill
FACULTÉ DES SCIENCES\\
UNIVERSITÉ DE SHERBROOKE\\  \vspace{1.5cm}
\vfill
Sherbrooke, Québec, [mois et année du dépôt final]
}

\end{center}


%------------------PAGE DE COMPOSITION DU JURY ----------------------------

\blankpage
\singlespacing
\begin{center}
%\vglue 1cm
Le [date du dépôt final]\\   \vspace{1cm} % à changer
\textit{Le jury a accepté le mémoire de Willian Vieira dans sa version finale.}\\  \vspace{1cm}
Membres du jury\\  \vspace{1cm}

Professeur Dominique Gravel\\
Directeur de recherche\\
Département de Biologie\\
Université de Sherbrooke\\ \vspace{13mm}

Professeur Robert L. Bradley\\
Codirecteur de recherche\\
Département de Biologie\\
Université de Sherbrooke\\ \vspace{13mm}

Professeur Guillaume Blanchet \\
Évaluateur interne\\
Département de Biologie\\
Université de Sherbrooke\\ \vspace{13mm}

Professeur William Godsoe\\
Évaluateur externe\\
BioProtection Research Centre\\
Lincoln University, Canterbury, New Zealand\\ \vspace{13mm}

Professeur Mark Vellend\\
Président-rapporteur\\
Département de biologie\\
Université de Sherbrooke\\ \vspace{13mm}
\end{center}



%------------------PREMIER DÉPÔT ----------------------------
% \linenumbers 	% ajoute des numero de ligne (pour PREMIER DÉPÔT SEULEMENT - COMMENTER POUR DÉPÔT FINAL)


%------------------REMERCIEMENTS ----------------------------
\blankpage
%\onehalfspacing
\setstretch{1.3} 		% un peu plus que 1.5 pour que ça concorde avec Word
\chapter*{REMERCIEMENTS} 
Bon, c'est finalement la fin. Je vais écris ce remerciement en français des rues, comme on dit au Brésil. En écrivant comme je parle, sans attention à la grammaire oppressive.
De toute manières ceci c'est le premier draft de mon remerciement. Après sept ans de these, je ne pourrais pas prendre simplement la derniere journee pour faire le remerciement. Il a beacoup de monde spéciale qui ont passé dans ma vie dont je dois remercie.

D'abord, je remercie Dominique et Robert de m'avoir accepté à developper ce projet. Ça a tout simplement changé ma vie.
Un merci spécial à Dominique qui m'a laissé la liberté d'explorer des plus divers sujet pendant ma thèse, de l'ecologie, des la stat, jusqu'à la programmation dont sera ma profession pour l'avenir.
Un merci aussi aux differents collaboration qui ont eu leur contribution à ma thèse. Merci à Isabelle Boulanger, Bill Shipley, Danile Houle et Dominique Arsenault. Merci aussi à mes examinateur Mark Vellend et William Godsoe d'avoir accepté notre invitation.

Je remercie mes ami.e.s qui ont vecu leur thèse en parallèle de la mienne, sans eux je n'aurais jamais pu completer ce projet.
Merci Antoine, Amaël et Az pour votre amitié, présence, support, et inspirantion dès ma premère journée au Quebec.Merci Andrew de ta présence à la fin de mon doctorat. Tes contribution sur cette thèse et sur ma formation sont major. Mais surtout merci de m'avoir récupéré d'un trou dont je ne pouvais voir aucune lumière.
Merci aussi à toutes les ami.e.s et collègues qui ont fait la vie dans le labo plus humaine. Un obrigado especial à Ben et Vinc de m'avoir laissé gagner à Mario Kart quelques fois.

Ce simple choix de venir faire une thèse dans une autre pays m'a fait connaître des gens extraordinaires.
Matchiudi, merci de partager cette aventure de doctorat avec moi, ta résilience est ma plus grand inspiration. Je ne pourrais jamais assez remercie ton support depuis tout ces années. Merci de faire partie de ma vie et contribuer dans la personne qui je suis aujourd'hui. À mes colocs et partaines, Antoine, Steve, et dernièrement Claire, François, Naty, et chaïchai pour les bons moments et support dans les mauvais. Merci spécial à Steve qui perdu tout sont week-end à regler des bugs latex pour que je puisse soumetre ma these à temps, et Claire pur la révision du français.

Merci à mes parents, mes frères et ma sœur pour support même à cette longue distance. Vous êtes depuis toujours une ma source d'inspiration. Merci aussi à mes parents enprunté, Corinne et Remi, qui ont été toujours present. Merci aussi à Steve, Ben, et Pier Luc pour les incontables soirée de jeux pendant la pandemie. Merci finalement à gang de la capoira pour me permetre de renforcer mes racines.

Enfin, comme nous sommes construits parmi nous exachnge avec le monde, je ramènera un peu de vous tous avec moi pour toujours.


%------------------SOMMAIRE ----------------------------
%\parskip 1.2in % added by me
\blankpage	% commenter si vous ne souhaitez pas une page impaire ici

\pagenumbering{roman}	% en chiffre romain
\chapter*{SUMMARY}
Climate change poses a pressing challenge for several species, particularly forest trees that are failing to follow temperature warming. To mitigate this impact and sustain forest ecosystems, we must understand the mechanisms driving their dynamics and distribution. In this thesis, we explore different methods for investigating forest dynamics and understanding their relationship to climate, competition, and forest management. In the first Chapter, we set this theoretical background.

In the second chapter, we extend a forest community state model derived from the metapopulation theory to formulate how forest management can accelerate the response of the boreal-temperate ecotone under warming temperatures. Two management practices effectively reduced colonization credit and increased range shift under temperature warming scenarios. While these results suggest that forest management could help forests keep pace with climate change at the community scale, we miss the local dynamics at the species level.

My third chapter addresses this issue by developing a species-specific size-structured integral projection model (IPM). Using forest inventories across the US and Quebec, we model growth, survival, and recruitment functions dependent on climate and competition to predict the stand-level asymptotic population growth rate ($\lambda$) of 31 tree species. We found that $\lambda$ was more sensitive to temperature than competition for all species, and the relative importance of climate increased at the borders of species located at the extreme climate ranges. These findings provide important insides on how species might respond to novel conditions arising from climate change, perturbations, and forest management. Yet, considerable uncertainty arising from local plot variability dominated the predictions.

In the fourth chapter, we explicitly account for the uncertainty and variability arising from various sources in the IPM to predict species performance in a probabilistic framework. Introducing a novel metric, local suitable probability, we quantified the average effect of climate and competition along with their spatiotemporal variation. We find that both climate and competition can determine range limits, but the climate was predominantly more effective in reducing suitable probability toward the species border. We finish this by discussing how suitable probability can link individual demographic rates and metapopulation dynamics.

In summary, these three chapters show the importance of considering multiple scales to gain a comprehensive understanding of forest dynamics. By integrating uncertainty across different scales, we were able to capture how climate and competition, along with their variability, influence the population-level performance of tree species. Furthermore, because our approach naturally propagates the uncertainty from the individual processes up to the population level, we can now track and quantify the exact source of variation in tree performance across their range. Based on these results, we propose a novel theory to reconcile the individual demographic rates with the metapopulation dynamics. This integrative approach allows one to account for both the local and landscape drivers of forest dynamics when assessing tree distribution.

\textbf{keywords:} demography, population growth rate, range dynamics, species distribution

\chapter*{SOMMAIRE}
\setcounter{page}{10} 		% ATTN spécifier manuellement où commence la numérotation des pages APRÈS LES REMERCIEMENTS et SELON LES PAGES BLANCHES
Les changements climatiques posent un défi urgent pour de nombreuses espèces. C'est particulièrement le cas pour des espèces telles que les arbres forestiers qui réagissent peu au réchauffement climatique. Pour atténuer cet impact et maintenir les écosystèmes forestiers, nous devons comprendre les mécanismes qui déterminent leurs dynamiques et leurs répartitions. Dans cette thèse, nous explorons différentes méthodes pour étudier la dynamique forestière et comprendre sa relation avec le climat, la compétition et la gestion forestière. 
Dans le premier chapitre, nous posons le contexte théorique indispensable à la bonne compréhension des chapitres suivants.

Dans le deuxième chapitre, nous présentons un modèle d'état de communauté forestière dérivé de la théorie des métapopulations pour évaluer si la gestion forestière pourrait accélérer la réponse de l'écotone boréal-tempéré au réchauffement des températures. Nous montrons que deux pratiques de gestion réduisent de façon significative le crédit de colonisation et augmentent le déplacement des aires de répartition des espèces vers le Nord dans des scénarios de réchauffement de la température. Bien que ces résultats suggèrent que la gestion forestière pourrait aider les forêts à suivre le rythme du changement climatique à l'échelle communautaire, cet approche ne prend pas en compte les dynamiques locales au niveau des espèces.

Mon troisième chapitre aborde cette question en développant un Modèle de Projection Intégrale (IPM) structuré en taille spécifique des espèces. À l'aide d'inventaires forestiers aux États-Unis et au Québec, nous modélisons les fonctions de croissance, de survie et de recrutement en fonction du climat et de la compétition pour prédire le taux de croissance asymptotique de la population ($\lambda$) au sein des peuplements de 31 espèces d'arbres. Nous constatons que $\lambda$ est plus sensible à la température qu'à la compétition pour toutes les espèces, et que l'importance relative du climat augmente aux frontières des aires de distribution espèces situées dans les plages climatiques extrêmes. Ces résultats fournissent des informations importantes sur la manière dont les espèces pourraient réagir aux nouvelles conditions résultant du changement climatique, des perturbations et de la gestion forestière. Pourtant, une incertitude considérable découlant de la variabilité locale des parcelles dominent les prédictions.

Dans le quatrième chapitre, nous tenons explicitement compte de l'incertitude et de la variabilité découlant de diverses sources de l'IPM pour prédire la performance des espèces dans un cadre probabiliste. En introduisant une nouvelle métrique, la probabilité de l'habitat local approprié, nous quantifions l'effet moyen du climat et de la compétition ainsi que leur variation spatio-temporelle. Nous constatons que le climat et la compétition peuvent déterminer les limites de l'aire de répartition, mais que le climat est principalement plus efficace pour réduire la probabilité de l'habitat local approprié vers la bordure de l'aire de répartition des espèces. Nous terminons en discutant comment une probabilité d'habitat local approprié peut lier les taux démographiques individuels et la dynamique des métapopulations.

Ces trois chapitres montrent l'importance de considérer plusieurs échelles organisationnelles pour améliorer notre compréhension globale de la dynamique forestière. En intégrant l'incertitude à ces différentes échelles, nous pouvons comprendre comment le climat et la compétition, ainsi que leur variabilité, influencent les performances des espèces d'arbres au niveau de la population. De plus, comme notre approche propage l'incertitude à l'échelle des processus individuels jusqu'au niveau de la population, nous pouvons désormais suivre et quantifier la source exacte de variation des performances des arbres dans leur aire de répartition. Sur la base de ces résultats, nous proposons une nouvelle théorie pour réconcilier les taux démographiques individuels avec la dynamique des métapopulations. Cette approche intégrative permet de prendre en compte à la fois les facteurs locaux et paysagers de la dynamique forestière lors de l'évaluation de la répartition des arbres.

\textbf{Mots clés :} démographie, taux de croissance de la population, dynamique de l'aire de répartition, répartition des espèces



%\blankpage			% decommenter si table des matières est sur page impaire


%------------------INCLURE TABLE DES MATIÈRES ET FIGURES ----------------------------
{
\setlength{\parskip}{0ex}
\cleardoublepage
\phantomsection 	% INCLUT TABLE DES MATIÈRES, SANS "TABLE DES MATIÈRES" DEDANS
\tableofcontents
}
\cleardoublepage
\listoftables

\cleardoublepage
\listoffigures


%------------------SIGLES ET ABRÉVIATIONS ----------------------------
\cleardoublepage
\chapter*{LISTE DES ABRÉVIATIONS ET DES SIGLES}

\begin{tabular}{ ll } 
 GPP & [Définition de l’abréviation] \\ 
 MHM & [Définition du sigle] \\ 
 \end{tabular}

 

%------------------CITATIONS----------------------------

\renewenvironment{quote}
  {\singlespacing\small\list{}{\rightmargin=2.5cm \leftmargin=2.5cm}%
   \item\relax}
  {\endlist}
  
  
%------------------CORPS DU DOCUMENT ----------------------------

\pagenumbering{arabic}
\mainmatter
%\onehalfspacing
%\setstretch{1.3}\include{chapter1/} 		% ATTENTION "\INPUT" GÉNÈRE LA BIBLIOGRAPHIE COMMUNE À L'INTRODUCTION ET LA CONCLUSION À LA FIN DE LA THÈSE. SINON  								 "\INCLUDE" GÉNÈRE PLUTÔT UNE BIBLIOGRAPHIE À LA FIN DE L'INTRO ET DE LA CONCLUSION. 
					% SI LA THÈSE EST CLASSIQUE IL N'EST PAS NÉCESSAIRE D'AVOIR UNE LISTE DE RÉFÉRENCES POUR CHAQUE CHAPITRE NUMÉROTÉ
\setstretch{1.3}\input{introduction}
\setstretch{1.3}\include{chapter1/manuscript_thesis}
\setstretch{1.3}\include{chapter2/manuscript_thesis}
\setstretch{1.3}\include{chapter3/manuscript_thesis}
\setstretch{1.3}\input{discussion}


%-------------------------------------------------------------------------------
%  ANNEXE A : Pour l'enlever, placer la ligne suivante en commentaire
%-------------------------------------------------------------------------------

\appendix
\renewcommand\chapterstring{ANNEXE}
% \addcontentsline{toc}{chapter}{ANNEXE}
\setstretch{1.3}\include{chapter1/suppInfo_thesis}
\setstretch{1.3}\include{chapter2/suppInfo_thesis}
\setstretch{1.3}\include{chapter3/suppInfo_thesis}


%------------------BIBLIOGRAPHIE GÉNÉRALE ----------------------------
% UTILISER LA LIGNE DE COMMANDE POUR COMPILER BIBTEX ET LATEX DANS LE BON ORDRE 

\singlespacing
\addcontentsline{toc}{chapter}{BIBLIOGRAPHIE}
{\renewcommand{\bibname}{BIBLIOGRAPHIE}
\renewcommand{\bibsection}{\chapter*{\bibname}}
\bibliography{library}}
\bibliographystyle{styles/myharvard}


%------------------FIN DU DOCUMENT ----------------------------
% add blank page after
\nolinenumbers
\blankpage
\blankpage
\end{document}