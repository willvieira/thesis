Les changements climatiques posent un défi urgent pour de nombreuses espèces. C'est particulièrement le cas pour des espèces telles que les arbres forestiers qui réagissent peu au réchauffement climatique. Pour atténuer cet impact et maintenir les écosystèmes forestiers, nous devons comprendre les mécanismes qui déterminent leurs dynamiques et leurs répartitions. Dans cette thèse, nous explorons différentes méthodes pour étudier la dynamique forestière et comprendre sa relation avec le climat, la compétition et la gestion forestière. 
Dans le premier chapitre, nous posons le contexte théorique indispensable à la bonne compréhension des chapitres suivants.

Dans le deuxième chapitre, nous présentons un modèle d'état de communauté forestière dérivé de la théorie des métapopulations pour évaluer si la gestion forestière pourrait accélérer la réponse de l'écotone boréal-tempéré au réchauffement des températures. Nous montrons que deux pratiques de gestion réduisent de façon significative le crédit de colonisation et augmentent le déplacement des aires de répartition des espèces vers le Nord dans des scénarios de réchauffement de la température. Bien que ces résultats suggèrent que la gestion forestière pourrait aider les forêts à suivre le rythme du changement climatique à l'échelle communautaire, cet approche ne prend pas en compte les dynamiques locales au niveau des espèces.

Mon troisième chapitre aborde cette question en développant un Modèle de Projection Intégrale (IPM) structuré en taille spécifique des espèces. À l'aide d'inventaires forestiers aux États-Unis et au Québec, nous modélisons les fonctions de croissance, de survie et de recrutement en fonction du climat et de la compétition pour prédire le taux de croissance asymptotique de la population ($\lambda$) au sein des peuplements de 31 espèces d'arbres. Nous constatons que $\lambda$ est plus sensible à la température qu'à la compétition pour toutes les espèces, et que l'importance relative du climat augmente aux frontières des aires de distribution espèces situées dans les plages climatiques extrêmes. Ces résultats fournissent des informations importantes sur la manière dont les espèces pourraient réagir aux nouvelles conditions résultant du changement climatique, des perturbations et de la gestion forestière. Pourtant, une incertitude considérable découlant de la variabilité locale des parcelles dominent les prédictions.

Dans le quatrième chapitre, nous tenons explicitement compte de l'incertitude et de la variabilité découlant de diverses sources de l'IPM pour prédire la performance des espèces dans un cadre probabiliste. En introduisant une nouvelle métrique, la probabilité de l'habitat local approprié, nous quantifions l'effet moyen du climat et de la compétition ainsi que leur variation spatio-temporelle. Nous constatons que le climat et la compétition peuvent déterminer les limites de l'aire de répartition, mais que le climat est principalement plus efficace pour réduire la probabilité de l'habitat local approprié vers la bordure de l'aire de répartition des espèces. Nous terminons en discutant comment une probabilité d'habitat local approprié peut lier les taux démographiques individuels et la dynamique des métapopulations.

Ces trois chapitres montrent l'importance de considérer plusieurs échelles organisationnelles pour améliorer notre compréhension globale de la dynamique forestière. En intégrant l'incertitude à ces différentes échelles, nous pouvons comprendre comment le climat et la compétition, ainsi que leur variabilité, influencent les performances des espèces d'arbres au niveau de la population. De plus, comme notre approche propage l'incertitude à l'échelle des processus individuels jusqu'au niveau de la population, nous pouvons désormais suivre et quantifier la source exacte de variation des performances des arbres dans leur aire de répartition. Sur la base de ces résultats, nous proposons une nouvelle théorie pour réconcilier les taux démographiques individuels avec la dynamique des métapopulations. Cette approche intégrative permet de prendre en compte à la fois les facteurs locaux et paysagers de la dynamique forestière lors de l'évaluation de la répartition des arbres.

\textbf{Mots clés :} démographie, taux de croissance de la population, dynamique de l'aire de répartition, répartition des espèces


