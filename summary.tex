Climate change poses a pressing challenge for several species, particularly forest trees that are failing to follow temperature warming. To mitigate this impact and sustain forest ecosystems, we must understand the mechanisms driving their dynamics and distribution. In this thesis, we explore different methods for investigating forest dynamics and understanding their relationship to climate, competition, and forest management. In the first Chapter, we set this theoretical background.

In the second chapter, we extend a forest community state model derived from the metapopulation theory to formulate how forest management can accelerate the response of the boreal-temperate ecotone under warming temperatures. Two management practices effectively reduced colonization credit and increased range shift under temperature warming scenarios. While these results suggest that forest management could help forests keep pace with climate change at the community scale, we miss the local dynamics at the species level.

My third chapter addresses this issue by developing a species-specific size-structured integral projection model (IPM). Using forest inventories across the US and Quebec, we model growth, survival, and recruitment functions dependent on climate and competition to predict the stand-level asymptotic population growth rate ($\lambda$) of 31 tree species. We found that $\lambda$ was more sensitive to temperature than competition for all species, and the relative importance of climate increased at the borders of species located at the extreme climate ranges. These findings provide important insides on how species might respond to novel conditions arising from climate change, perturbations, and forest management. Yet, considerable uncertainty arising from local plot variability dominated the predictions.

In the fourth chapter, we explicitly account for the uncertainty and variability arising from various sources in the IPM to predict species performance in a probabilistic framework. Introducing a novel metric, local suitable probability, we quantified the average effect of climate and competition along with their spatiotemporal variation. We find that both climate and competition can determine range limits, but the climate was predominantly more effective in reducing suitable probability toward the species border. We finish this by discussing how suitable probability can link individual demographic rates and metapopulation dynamics.

In summary, these three chapters show the importance of considering multiple scales to gain a comprehensive understanding of forest dynamics. By integrating uncertainty across different scales, we were able to capture how climate and competition, along with their variability, influence the population-level performance of tree species. Furthermore, because our approach naturally propagates the uncertainty from the individual processes up to the population level, we can now track and quantify the exact source of variation in tree performance across their range. Based on these results, we propose a novel theory to reconcile the individual demographic rates with the metapopulation dynamics. This integrative approach allows one to account for both the local and landscape drivers of forest dynamics when assessing tree distribution.

\textbf{keywords:} demography, population growth rate, range dynamics, species distribution
